% Acronyms

\newacronym{css}{CSS}{Cascading Style Sheets}
\newacronym{dom}{DOM}{Document Object Model}
\newacronym{js}{JS}{JavaScript}
\newacronym{html}{HTML}{HyperText Markup Language}

\newacronym{ui}{UI}{User Interface}
\newacronym{aria}{ARIA}{Accessible Rich Internet Applications}
\newacronym{wcag}{WCAG}{Web Content Accessibility Guidelines}
\newacronym{mcag}{MCAG}{Mobile Content Accessibility Guidelines}
\newacronym[description={\glslink{apig}{\textit{Application Program Interface}}}]
    {api}{API}{\textit{Application Program Interface}}
\newacronym{loc}{LOC}{Lines of Code}

\newacronym[description={\glslink{iag}{\textit{Artificial Intelligence}}}]
    {ai}{AI}{\textit{Artificial Intelligence}}

\newacronym[description={\glslink{iag}{Intelligenza Artificiale}}]
    {ia}{IA}{Intelligenza Artificiale}
\newacronym[description={\glslink{llmg}{\textit{Large Language Model}}}]
    {llm}{LLM}{\textit{Large Language Model}}
\newacronym[description={\glslink{w3cg}{\textit{World Wide Web Consortium}}}]
    {w3c}{W3C}{\textit{World Wide Web Consortium}}
\newacronym[description={\glslink{xmlg}{\textit{Extensible Markup Language}}}]
    {xml}{XML}{\textit{Extensible Markup Language}}
\newacronym[description={\glslink{urlg}{\textit{Uniform Resource Locator}}}]
    {url}{URL}{\textit{Uniform Resource Locator}}
\newacronym[description={\glslink{gpug}{\textit{Graphics Processing Unit}}}]
    {gpu}{GPU}{\textit{Graphics Processing Unit}}
\newacronym{mv2}{MV2}{Manifest V2}
\newacronym{mv3}{MV3}{Manifest V3}
\newacronym{poc}{PoC}{\glslink{pocg}{\textit{Proof of Concept}}}
\newacronym{tcp}{TCP}{\glslink{tcpg}{\textit{Transmission Control Protocol}}}
\newacronym{tv}{TV}{Total Validator}

% Glossary

\newglossaryentry{apig} {
    name=\acrshort{api},
    text=Application Program Interface,
    sort=api,
    description={in informatica con il termine \emph{Application Programming Interface} (ing. interfaccia di programmazione di un'applicazione) si indicano regole e specifiche per la comunicazione tra \textit{software}. \\
    Tali regole fungono da interfaccia tra i vari \textit{software} e ne facilitano l’interazione, allo stesso modo in cui l’interfaccia utente facilita l’interazione tra uomo e \textit{computer}. \\
    \cite{site:api-treccani}
    }
}

\newglossaryentry{aig} {
    name=\acrshort{ia},
    text=Artificial Intelligence,
    sort=ai,
    description={insieme di studi e tecniche, pertinenti all’informatica, ma prossime alle ricerche di logica matematica e con profonde implicazioni sia filosofiche sia sociali, che mirano alla realizzazione di macchine o programmi in grado di risolvere problemi e di riprodurre attività proprie dell’intelligenza umana o che comunque ne simulino il comportamento. \\
    \cite{site:ai-treccani}
    }
}

\newglossaryentry{domg}{ 
    name={\acrshort{dom}},
    text={Document Object Model},
    sort=dom,
    description={in informatica il Document Object Model (spesso abbreviato come \acrshort{dom}), lett. "modello a oggetti del documento", è una forma di rappresentazione dei documenti strutturati come modello orientato agli oggetti. È lo standard ufficiale del \acrshort{w3c} per la rappresentazione di documenti strutturati in maniera da essere neutrali sia per la lingua che per la piattaforma. \\
    \cite{site: dom-wiki}}
}

\newglossaryentry{backendg}{
    name=\textit{Backend},
    text=\textit{Backend},
    sort=back-end,
    description= {con il termine \textit{"backend"} si intende la parte non visibile all'utente di un programma, che elabora e gestisce i dati generati dall'interfaccia grafica.\\
    \cite{site:backend-wiki}
    }
}

\newglossaryentry{llmg}{
    name=\acrshort{llm},
    text={Large Language Model},
    sort=llm,
    description= {con il termine \textit{"Large Language Model"} si intende un algoritmo di \glslink{iag}{intelligenza artificiale} che, processando massivamente una grande quantità di dati, utilizza tecniche di \textit{deep learning} in vari ambiti dell’elaborazione del linguaggio naturale come la comprensione, la traduzione, la generazione e la previsione di nuovi contenuti.\\
    \cite{site:llm-treccani}
    }
}


\newglossaryentry{w3cg}{
    name=\acrshort{w3c},
    text={W3C},
    sort=w3c,
    description= {il \textit{World Wide Web Consortium}, anche conosciuto come \acrshort{w3c}, è un'organizzazione non governativa internazionale che ha come scopo quello di favorire lo sviluppo di tutte le potenzialità del \glslink{wwwg}{World Wide Web} e diffondere la cultura dell'accessibilità della Rete.\\
    \cite{site:w3c-wiki}
    }
}

\newglossaryentry{wwwg}{
    name=\textit{World Wide Web},
    text={World Wide Web},
    sort=world-wide-web,
    description= {il \textit{World Wide Web} (termine in lingua inglese traducibile in italiano come "rete di ampiezza mondiale" o "rete mondiale"), è uno dei principali servizi di Internet, che permette di navigare e usufruire di un insieme molto vasto di contenuti, multimediali e non, interrelati tramite collegamenti ipertestuali (\textit{link}), e di fruire di ulteriori servizi accessibili a tutti o ad una parte selezionata degli utenti di Internet (mediante autenticazione).\\
    \cite{site:www-wiki}
    }
}

\newglossaryentry{xmlg}{
    name=\textit{eXtensible Markup Language},
    text={eXtensible Markup Language},
    sort=extensible-markup-language,
    description= {in informatica, l'\acrshort{xml} (sigla di \textit{eXtensible Markup Language}, lett. "linguaggio di marcatura estendibile") è un metalinguaggio per la definizione di linguaggi di \glslink{markupg}{markup}, ovvero un linguaggio basato su un meccanismo sintattico che consente di definire e controllare il significato degli elementi contenuti in un documento o in un testo.\\
    \cite{site:xml-wiki}
    }
}

\newglossaryentry{urlg}{
    name=\acrshort{url},
    text={Uniform Resource Locator},
    sort=url,
    description= {nel linguaggio informatico, sigla dell’ingl.\textit{Uniform Resource Locator} «localizzatore unico della risorsa (informatica)», indirizzo di un sito web espresso in modo univoco e con una forma utilizzabile dal \glslink{browserg}{browser}.\\
    \cite{site:url-treccani}
    }
}

\newglossaryentry{gpug}{
    name=\acrshort{gpu},
    text={Graphics Processing Unit},
    sort=gpu,
    description= {l'unità di elaborazione grafica (in acronimo \acrshort{gpu}, dall'inglese \textit{Graphics Processing Unit}), è un processore progettato per accelerare la creazione di immagini in un frame buffer, destinato all'output su un dispositivo di visualizzazione. \\
     \cite{site:gpu-wiki}
    }
}

\newglossaryentry{chatbotg}{
    name=\textit{Chatbot},
    text={Chatbot},
    sort=chatbot,
    description= {è un software progettato per simulare una conversazione con un essere umano. \\
     \cite{site:chatbot-wiki}
    }
}

\newglossaryentry{pluging}{
    name=\textit{Plugin},
    text={Plugin},
    sort=plugin,
    description= {in campo informatico è un programma non autonomo che interagisce con un altro programma per ampliarne o estenderne le funzionalità originarie. \\
     \cite{site:plugin-wiki}
    }
}

\newglossaryentry{markupg}{
    name=\textit{Markup},
    text={Markup},
    sort=markup,
    description= {un linguaggio di \textit{markup} (in italiano linguaggio di marcatura o linguaggio di formattazione) è un insieme di regole che descrivono i meccanismi di rappresentazione (strutturali, semantici, presentazionali) o d'impaginazione di un testo. \\
     \cite{site:markup-wiki}
    }
}

\newglossaryentry{formg}{
    name=\textit{Form},
    text={Form},
    sort=form,
    description= {in Internet, modulo telematico a disposizione dell’utente per l’inserimento e l’invio di dati. \\
     \cite{site:form-treccani}
    }
}

\newglossaryentry{browserg}{
    name=\textit{Browser},
    text={Browser},
    sort=browser,
    description= {nel linguaggio informatico, programma di un computer che permette il collegamento alla rete Internet e mediante il quale si può navigare da un sito telematico all’altro. \\
     \cite{site:browser-treccani}
    }
}

\newglossaryentry{scriptg}{
    name=\textit{Script},
    text={Script},
    sort=script,
    description= {in informatica, programma o sequenza di istruzioni che viene interpretata o portata a termine da un altro programma. \\
     \cite{site:script-treccani}
    }
}

\newglossaryentry{asincronog}{
    name=Asincrono,
    text={Asincrono},
    sort=asincrono,
    description= {non sincrono, che non avviene o si manifesta cioè nel medesimo tempo, o, in senso più tecnico, che manca di sincronismo. \\
     \cite{site:asincrono-treccani}
    }
}

\newglossaryentry{client-sideg}{
    name=\textit{Client-side},
    text={Client-side},
    sort=client-side,
    description= {in informatica, nell'ambito delle reti di calcolatori, il termine lato client (client-side in inglese) indica le operazioni di elaborazione effettuate da un client in un'architettura client-server. \\
     \cite{site:client_side-wiki}
    }
}

\newglossaryentry{serverg}{
    name=\textit{Server},
    text={Server},
    sort=server,
    description= {in informatica (con riferimento a una rete di calcolatori), calcolatore che svolge funzioni di servizio per tutti i calcolatori collegati. \\
     \cite{site:server-treccani}
    }
}

\newglossaryentry{cloud-computingg}{
    name=\textit{Cloud computing},
    text={Cloud computing},
    sort=cloud-computing,
    description= {la tecnologia che, sotto forma di servizio offerto dal provider al cliente, permette di memorizzare e di elaborare i dati e i programmi di un utente grazie all’utilizzo di risorse hardware o software distribuite in rete. \\
     \cite{site:cloud_computing-treccani}
    }
}

\newglossaryentry{latenzag}{
    name=Latenza,
    text={Latenza},
    sort=latenza,
    description= {in informatica, il tempo impiegato da un’informazione per andare da un’unità all’altra di un sistema, in partic. da un sensore al relativo elaboratore (è detto anche l. di risposta). \\
     \cite{site:latenza-treccani}
    }
}

\newglossaryentry{quantizzazioneg}{
    name=Quantizzazione,
    text={Quantizzazione},
    sort=quantizzazione,
    description= {in elettronica e nella tecnica delle telecomunicazioni, l’operazione, attuata con un quantizzatore, consistente nel suddividere il campo di variabilità di una grandezza continua in un numero finito di intervalli (definiti a volte «quanti»), in ciascuno dei quali la grandezza è considerata costante e sostituita con un valore rappresentativo. \\
     \cite{site:quantizzazione-treccani}
    }
}

\newglossaryentry{open-sourceg}{
    name=\textit{Open source},
    text={Open source},
    sort=open-source,
    description= {in informatica, software non protetto da copyright, il cui codice sorgente è lasciato alla disponibilità degli utenti e quindi liberamente modificabile. \\
     \cite{site:open_source-treccani}
    }
}

\newglossaryentry{debuggingg}{
    name=\textit{Debugging},
    text={Debugging},
    sort=debugging,
    description= {nel linguaggio dell’informatica, operazione di messa a punto di un programma, un’applicazione, ecc., consistente nella ricerca (di norma effettuata dall’elaboratore) e nella correzione (talvolta automatica) degli errori di procedura, relativi al tipo di linguaggio impiegato, che impediscono o rendono difettosa l’elaborazione. \\
     \cite{site:debugging-treccani}
    }
}

\newglossaryentry{desktop-firstg}{
    name=\textit{Desktop-first},
    text={Desktop-first},
    sort=desktop-first,
    description= { significa progettare il sito web avendo come obiettivo principale l'esperienza su desktop. Sebbene sarebbe facile pensare che il desktop-first sia un approccio ormai superato, in realtà esistono diverse ragioni per cui molti scelgono ancora di partire in grande e poi ridimensionare. \\
     \cite{site:desktop_first-hvdig}
    }
}

\newglossaryentry{iag}{
    name=\textit{Intelligenza Artificiale},
    text={Intelligenza artificiale},
    sort=intelligenza-artificiale,
    description= { insieme di studi e tecniche, pertinenti all’informatica, ma prossime alle ricerche di logica matematica e con profonde implicazioni sia filosofiche sia sociali, che mirano alla realizzazione di macchine o programmi in grado di risolvere problemi e di riprodurre attività proprie dell’intelligenza umana o che comunque ne simulino il comportamento. \\
     \cite{site:ai-treccani}
    }
}

\newglossaryentry{ChatGPTg} {
    name=\textit{ChatGPT},
    text=ChatGPT,
    sort=ChatGPT,
    description={ implementazione del modello di intelligenza artificiale (\acrshort{ia}) sviluppato da OpenAI, basato sull'architettura GPT-3 (Generative Pre-trained Transformer 3) e progettato per comprendere e generare testo in linguaggio naturale. \\
    \cite{site:chatgpt-treccani}
    }
}

\newglossaryentry{Chrome-basedg} {
    name=\textit{Chrome-based},
    text=Chrome-based,
    sort=Chrome-based,
    description={ un \glslink{browserg}{browser} o software che utilizza la base del codice sorgente del progetto \glslink{open-sourceg}{open-source} Chromium, la stessa foundation di Google Chrome.
    }
}

\newglossaryentry{gitg} {
    name=\textit{GIT},
    text=GIT,
    sort=GIT,
    description={ è un'applicazione software per il controllo di versione distribuito utilizzabile da interfaccia a riga di comando, creato da Linus Torvalds nel 2005. \\
    \cite{site:git-wiki}
    }
}

\newglossaryentry{pocg} {
    name=\acrshort{poc},
    text=Proof of Concept,
    sort=poc,
    description={ è una realizzazione incompleta o abbozzata (sinopsi) di un determinato progetto o metodo per dimostrare la fattibilità o confermare la validità di alcuni principi o concetti fondamentali. Un esempio tipico è quello di un prototipo. \\
    \cite{site:poc-wiki}
    }
}

\newglossaryentry{webassemblyg} {
    name=\textit{WebAssembly},
    text=WebAssembly,
    sort=webAssembly,
    description={ è uno standard web che definisce un formato binario e un corrispondente formato testuale per la scrittura di codice eseguibile nelle pagine web. Ha lo scopo di abilitare l'esecuzione del codice quasi alla stessa velocità con cui esegue il codice macchina nativo. \\
    \cite{site:webassembly-wiki}
    }
}

\newglossaryentry{websocketg} {
    name=\textit{WebSocket},
    text=WebSocket,
    sort=websocket,
    description={ è un protocollo che fornisce canali di comunicazione full-duplex (trasmissione bidirezionale simultanea) attraverso una singola connessione \acrshort{tcp}. \\
    \cite{site:websocket-wiki}
    }
}

\newglossaryentry{tcpg} {
    name=\acrshort{tcp},
    text=Transmission Control Protocol,
    sort=transmission-control-protocol,
    description={ è un protocollo di rete a pacchetto di livello di trasporto, appartenente alla suite di protocolli Internet, che si occupa di controllo della trasmissione ovvero rendere affidabile la comunicazione dati in rete tra mittente e destinatario. \\
    \cite{site:tcp-wiki}
    }
}

\newglossaryentry{open-weightg} {
    name=\textit{Open weight},
    text=Open weight,
    sort=open-weight,
    description={ i modelli a peso aperto sono sistemi di \glslink{iag}{intelligenza artificiale} in cui i “pesi” effettivi, i numeri fondamentali che il modello ha appreso durante l’addestramento, sono resi pubblici. Questi pesi sono ciò che guida le previsioni, le risposte e il comportamento generale del modello. \\
    \cite{site:open_weight-invezz}
    }
}
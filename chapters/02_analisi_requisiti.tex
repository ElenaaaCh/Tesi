\chapter{Analisi dei requisiti}
\label{chap:analisi-requisiti}
%\footnote{\cite{site:agile-manifesto}} per citare

\section{Obiettivo del progetto}
\label{sec:obiettivo}
\noindent L’obiettivo principale del progetto è la realizzazione di un’estensione web interattiva con chatbot integrato, pensata per supportare gli sviluppatori nell’individuazione e nella correzione degli errori di accessibilità presenti nelle pagine web. \\Il progetto si propone di rispondere alle seguenti esigenze:
\begin{itemize}
    \item Fornire un linguaggio chiaro e facilmente comprensibile, anche per chi non ha conoscenze approfondite delle linee guida di accessibilità;
    \item Offrire un supporto immediato e contestuale durante lo sviluppo del codice;
    \item Facilitare il miglioramento della qualità e dell’accessibilità dei siti web, promuovendo buone pratiche di progettazione inclusiva.
\end{itemize}

\section{Casi d'uso}
\label{sec:uc}

\section{Tracciamento dei requisiti}
\label{sec:req}
\noindent Dall’analisi dei requisiti e degli UC effettuata sono emersi dei requisiti di diverso tipo, ai quali è stato associato un codice identificativo per distinguerli. Si farà riferimento ai requisiti secondo la seguente classificazione:
\begin{itemize}
    \item \textbf{RO} per i requisiti obbligatori, vincolanti in quanto obiettivo primario richiesto dal committente;
    \item \textbf{RD} per i requisiti desiderabili, non vincolanti o strettamente necessari, ma dal riconoscibile valore aggiunto;
    \item \textbf{RF} per i requisiti facoltativi, rappresentanti valore aggiunto non strettamente competitivo.
\end{itemize}

\noindent Le sigle precedentemente indicate saranno seguite da un trattino e da una coppia sequenziale di numeri, per identificare il singolo requisito in maniera univoca.

\subsection{Requisiti obbligatori}
\begin{footnotesize}
\begin{longtable}[c]{|C{2.85cm}|C{8cm}|C{2.85cm}|}
\caption{Tabella del tracciamento dei requisiti obbligatori}
\label{tab:requisiti_obbligatori}\\
\hline
\textbf{Requisito} & \textbf{Descrizione} & \textbf{Fonti}\\
\hline
\endfirsthead
\multicolumn{3}{c}%
{{\bfseries Tabella \thetable\ -- continua dalla pagina precedente}} \\
\hline
\textbf{Requisito} & \textbf{Descrizione} & \textbf{Fonti}\\
\hline
\endhead
\hline
\multicolumn{3}{r}{{ -- continua nella pagina successiva}} \\
\endfoot
\hline
\endlastfoot
\textbf{RO-01} & xx & UC xx\\
\hline
\textbf{RO-02} & xx & UC xx\\
\hline
\end{longtable}
\end{footnotesize}

\subsection{Requisiti desiderabili}
\begin{footnotesize}
\begin{longtable}[c]{|C{2.85cm}|C{8cm}|C{2.85cm}|}
\caption{Tabella del tracciamento dei requisiti desiderabili}
\label{tab:requisiti_desiderabili}\\
\hline
\textbf{Requisito} & \textbf{Descrizione} & \textbf{Fonti}\\
\hline
\endfirsthead
\multicolumn{3}{c}%
{{\bfseries Tabella \thetable\ -- continua dalla pagina precedente}} \\
\hline
\textbf{Requisito} & \textbf{Descrizione} & \textbf{Fonti}\\
\hline
\endhead
\hline
\multicolumn{3}{r}{{ -- continua nella pagina successiva}} \\
\endfoot
\hline
\endlastfoot
\textbf{RD-01} & xx & UC xx\\
\hline
\textbf{RD-02} & xx & UC xx\\
\hline
\end{longtable}
\end{footnotesize}

\subsection{Requisiti facoltativi}
\begin{footnotesize}
\begin{longtable}[c]{|C{2.85cm}|C{8cm}|C{2.85cm}|}
\caption{Tabella del tracciamento dei requisiti facoltativi}
\label{tab:requisiti_facoltativi}\\
\hline
\textbf{Requisito} & \textbf{Descrizione} & \textbf{Fonti}\\
\hline
\endfirsthead
\multicolumn{3}{c}%
{{\bfseries Tabella \thetable\ -- continua dalla pagina precedente}} \\
\hline
\textbf{Requisito} & \textbf{Descrizione} & \textbf{Fonti}\\
\hline
\endhead
\hline
\multicolumn{3}{r}{{ -- continua nella pagina successiva}} \\
\endfoot
\hline
\endlastfoot
\textbf{RF-01} & xx & UC xx\\
\hline
\textbf{RF-02} & xx & UC xx\\
\hline
\end{longtable}
\end{footnotesize}



\chapter{Analisi dei requisiti}
\label{chap:analisi-requisiti}
%\footnote{\cite{site:agile-manifesto}} per citare

\section{Obiettivo del progetto}
\label{sec:obiettivo}
\noindent L’obiettivo principale del progetto è la realizzazione di un’estensione web interattiva con chatbot integrato, pensata per supportare gli sviluppatori nell’individuazione e nella correzione degli errori di accessibilità presenti nelle pagine web. \\Il progetto si propone di rispondere alle seguenti esigenze:
\begin{itemize}
    \item Fornire un linguaggio chiaro e facilmente comprensibile, anche per chi non ha conoscenze approfondite delle linee guida di accessibilità;
    \item Offrire un supporto immediato e contestuale durante lo sviluppo del codice;
    \item Facilitare il miglioramento della qualità e dell’accessibilità dei siti web, promuovendo buone pratiche di progettazione inclusiva.
\end{itemize}

\section{Casi d'uso}
\label{sec:uc}

\subsection{Attori}
\noindent Un attore rappresenta un elemento esterno al sistema, che può essere una persona o un altro sistema software, che interagisce con esso per il raggiungimento di un determinato obiettivo. L’attore non coincide quindi con un singolo utente specifico, ma con una categoria di utenti che condividono lo stesso ruolo nell’utilizzo del sistema.\\
Nel caso dell’estensione \textit{SviluppAbile}, l’attore identificato è lo sviluppatore web, interessato a migliorare la qualità del codice prodotto e a garantire che i siti realizzati siano accessibili ad un pubblico ampio e diversificato.

\subsection{Elenco e descrizione dei casi d'uso}
\subsubsection*{UC0: Visualizzazione prompt iniziale}
\addcontentsline{toc}{paragraph}{UC0: Visualizzazione prompt iniziale}
\noindent \textbf{Attori principali}: Sviluppatore web.\\
\textbf{Precondizioni}: L'utente ha avviato l’estensione.\\
\textbf{Descrizione}: L’estensione offre la possibilità di scegliere tra l’analisi assistita e la modalità guidata.\\
\textbf{Postcondizioni}: L’estensione è pronta per visualizzare il risultato della modalità scelta.\\

\subsubsection*{UC1: Modalità analisi assistita}
\addcontentsline{toc}{paragraph}{UC1: Analisi assistita}
\noindent \textbf{Attori principali}: Sviluppatore web.\\
\textbf{Precondizioni}: L’utente ha selezionato la modalità di analisi assistita.\\
\textbf{Descrizione}: L’estensione esegue la scansione del DOM della pagina web e mostra il codice sul pannello a sinistra. Nel pannello a destra è visibile la chat vuota con due domande suggerite e un campo di testo per l'inserimento di domande.\\
\textbf{Postcondizioni}: L’utente visualizza il DOM e la chat che potrà utilizzare.\\

\subsubsection*{UC1.1: Utilizzo delle domande suggerite}
\noindent \textbf{Attori principali}: Sviluppatore web.\\
\textbf{Precondizioni}: L’utente ha avviato la modalità di analisi assistita.\\
\textbf{Descrizione}: L’utente clicca su una domanda suggerita e la invia all'IA.\\
\textbf{Postcondizioni}: L’utente riceve una risposta contestualizzata, vede eventuali righe pertinenti del DOM evidenziate e nella chat appaiono delle ulteriori domande suggerite correlate.

\subsubsection*{UC1.2: Inserimento domanda}
\noindent \textbf{Attori principali}: Sviluppatore web.\\
\textbf{Precondizioni}: L’utente ha aperto la chat in modalità di analisi assistita.\\
\textbf{Descrizione}: L’utente inserisce manualmente una domanda nel form dedicato della chat e la invia all'IA.\\
\textbf{Postcondizioni}: L’utente riceve una risposta contestualizzata, vede eventuali righe pertinenti del DOM evidenziate e nella chat appaiono delle ulteriori domande suggerite correlate.


\subsubsection*{UC2: Modalità sviluppo guidato}
\addcontentsline{toc}{paragraph}{UC2: Modalità sviluppo guidato}
\noindent \textbf{Attori principali}: Sviluppatore web.\\
\textbf{Precondizioni}: L’utente ha selezionato la modalità di sviluppo guidato.\\
\textbf{Descrizione}: L’estensione mostra il DOM della pagina, il pannello centrale con il codice generato e la chat interattiva. \\
\textbf{Postcondizioni}: L’utente dispone di uno spazio di sviluppo assistito in cui ricevere sia chiarimenti teorici sia frammenti di codice pronti per l’uso.\\

\subsubsection*{UC2.1: Utilizzo delle domande suggerite}
\noindent \textbf{Attori principali}: Sviluppatore web.\\
\textbf{Precondizioni}: L’utente ha avviato la chat in modalità sviluppo guidato.\\
\textbf{Descrizione}: L’utente clicca su una domanda suggerita e la invia all'IA.\\
\textbf{Postcondizioni}: L’utente riceve una risposta contestualizzata, con eventuale blocco di codice visualizzato nel pannello centrale e suggerimenti per approfondire l’analisi.\\

\subsubsection*{UC2.2: Inserimento domanda}
\noindent \textbf{Attori principali}: Sviluppatore web.\\
\textbf{Precondizioni}: L’utente ha aperto la chat in modalità di sviluppo guidato.\\
\textbf{Descrizione}: L’utente inserisce manualmente una domanda nel form dedicato della chat e la invia all'IA.\\
\textbf{Postcondizioni}: L’utente riceve una risposta contestualizzata, con eventuale blocco di codice visualizzato nel pannello centrale e suggerimenti per approfondire l’analisi.

\subsubsection*{UC2.2: Copia di un blocco di codice}
\noindent \textbf{Attori principali}: Sviluppatore web.\\
\textbf{Precondizioni}: L’IA ha generato almeno un blocco di codice in risposta a una domanda dell’utente.\\
\textbf{Descrizione}: L’estensione visualizza il blocco di codice numerato; l’utente può copiarlo negli appunti per utilizzarlo immediatamente nel proprio progetto.\\
\textbf{Postcondizioni}: Il frammento selezionato è disponibile negli appunti dell’utente.\\

\subsubsection*{UC2.3: Download del codice generato}
\noindent \textbf{Attori principali}: Sviluppatore web.\\
\textbf{Precondizioni}: L’estensione ha generato uno o più blocchi di codice in modalità guidata.\\
\textbf{Descrizione}: L’utente clicca i pulsante \textit{“Scarica codice”}; l’estensione concatena tutti i blocchi generati e produce un file HTML scaricabile.\\
\textbf{Postcondizioni}: L’utente ottiene un file contenente tutti i suggerimenti di codice prodotti dall’IA.\\


\subsubsection*{UC3: Blocco estensione su pagine di sistema}
\addcontentsline{toc}{paragraph}{UC3: Blocco estensione su pagine di sistema}
\noindent \textbf{Attori principali}: Sviluppatore web.\\
\textbf{Precondizioni}: L’utente tenta di avviare l’estensione in una pagina di sistema del browser (es. \texttt{chrome://}, \texttt{edge://}, \texttt{about:}, nuova scheda).\\
\textbf{Descrizione}: L’estensione rileva che la pagina corrente appartiene alle aree riservate del browser, non accessibili alle estensioni. L'estensione apre un popup informativo che comunica all’utente l’impossibilità di utilizzare \textit{SviluppAbile} in quella pagina.\\
\textbf{Postcondizioni}: L’utente è informato del blocco e comprende che l’estensione può essere utilizzata solo su pagine web ordinarie.\\

\subsubsection*{UC4: Gestione errori di comunicazione con l’IA}
\addcontentsline{toc}{paragraph}{UC4: Gestione errori di comunicazione con l’IA}
\noindent \textbf{Attori principali}: Sviluppatore web.\\
\textbf{Precondizioni}: L’utente ha inviato una richiesta all’IA tramite chat (in modalità analisi assistita o sviluppo guidato).\\
\textbf{Descrizione}: L’estensione gestisce risposte non valide o errori di parsing JSON provenienti dal server dell’IA notificando l’utente.\\
\textbf{Postcondizioni}: L’utente è consapevole dell’errore e può ripetere la richiesta o verificare la configurazione del sistema.\\


\section{Tracciamento dei requisiti}
\label{sec:req}
\noindent Dall’analisi dei requisiti e degli UC effettuata sono emersi dei requisiti di diverso tipo, ai quali è stato associato un codice identificativo per distinguerli. Si farà riferimento ai requisiti secondo la seguente classificazione:
\begin{itemize}
    \item \textbf{RO} per i requisiti obbligatori, vincolanti in quanto obiettivo primario richiesto dal committente;
    \item \textbf{RD} per i requisiti desiderabili, non vincolanti o strettamente necessari, ma dal riconoscibile valore aggiunto;
    \item \textbf{RF} per i requisiti facoltativi, rappresentanti valore aggiunto non strettamente competitivo.
\end{itemize}

\noindent Le sigle precedentemente indicate saranno seguite da un trattino e da una coppia sequenziale di numeri, per identificare il singolo requisito in maniera univoca.

\subsection{Requisiti obbligatori}
\begin{footnotesize}
\begin{longtable}[c]{|L{2.85cm}|L{8cm}|L{2.85cm}|}
\caption{Tabella del tracciamento dei requisiti obbligatori}
\label{tab:requisiti_obbligatori}\\
\hline
\textbf{Requisito} & \textbf{Descrizione} & \textbf{Fonti}\\
\hline
\endfirsthead
\multicolumn{3}{c}%
{{\bfseries Tabella \thetable\ -- continua dalla pagina precedente}} \\
\hline
\textbf{Requisito} & \textbf{Descrizione} & \textbf{Fonti}\\
\hline
\endhead
\hline
\multicolumn{3}{r}{{ -- continua nella pagina successiva}} \\
\endfoot
\hline
\endlastfoot
\textbf{RO-01} & L’estensione deve offrire un prompt iniziale che consenta all’utente di scegliere tra modalità analisi assistita e modalità sviluppo guidato. & UC0\\
\hline
\textbf{RO-02} & In modalità analisi assistita, l’estensione deve permettere di visualizzare il DOM numerato della pagina e di porre domande all’IA. & UC1\\
\hline
\textbf{RO-03} & In modalità sviluppo guidato, l’estensione deve mostrare il DOM numerato, il pannello centrale con i suggerimenti di codice e la chat laterale. & UC2\\
\hline
\textbf{RO-04} & L’estensione deve impedire l’avvio su pagine di sistema del browser (es. chrome://, about:blank, nuova scheda). & UC3\\
\hline
\textbf{RO-05} & L’estensione deve gestire eventuali errori di comunicazione con l’IA mostrando messaggi esplicativi all’utente. & UC4\\
\hline
\end{longtable}
\end{footnotesize}

\subsection{Requisiti desiderabili}
\begin{footnotesize}
\begin{longtable}[c]{|L{2.85cm}|L{8cm}|L{2.85cm}|}
\caption{Tabella del tracciamento dei requisiti desiderabili}
\label{tab:requisiti_desiderabili}\\
\hline
\textbf{Requisito} & \textbf{Descrizione} & \textbf{Fonti}\\
\hline
\endfirsthead
\multicolumn{3}{c}%
{{\bfseries Tabella \thetable\ -- continua dalla pagina precedente}} \\
\hline
\textbf{Requisito} & \textbf{Descrizione} & \textbf{Fonti}\\
\hline
\endhead
\hline
\multicolumn{3}{r}{{ -- continua nella pagina successiva}} \\
\endfoot
\hline
\endlastfoot
\textbf{RD-01} & L’estensione deve proporre domande suggerite per facilitare l’interazione con l’IA in entrambe le modalità. & UC1.1, UC2.1\\
\hline
\textbf{RD-02} & In modalità analisi assistita, l’estensione deve evidenziare nel codice le righe pertinenti alla risposta dell’IA. & UC1.2\\
\hline
\textbf{RD-03} & In modalità sviluppo guidato, l’IA deve generare suggerimenti di codice accessibile numerati e separati in blocchi. & UC2\\
\hline
\textbf{RD-04} & L’utente deve poter scaricare un file HTML contenente tutti i blocchi di codice generati. & UC2.3\\
\hline
\end{longtable}
\end{footnotesize}

\subsection{Requisiti facoltativi}
\begin{footnotesize}
\begin{longtable}[c]{|L{2.85cm}|L{8cm}|L{2.85cm}|}
\caption{Tabella del tracciamento dei requisiti facoltativi}
\label{tab:requisiti_facoltativi}\\
\hline
\textbf{Requisito} & \textbf{Descrizione} & \textbf{Fonti}\\
\hline
\endfirsthead
\multicolumn{3}{c}%
{{\bfseries Tabella \thetable\ -- continua dalla pagina precedente}} \\
\hline
\textbf{Requisito} & \textbf{Descrizione} & \textbf{Fonti}\\
\hline
\endhead
\hline
\multicolumn{3}{r}{{ -- continua nella pagina successiva}} \\
\endfoot
\hline
\endlastfoot
\textbf{RF-01} & L’estensione deve supportare una modalità “notte” per migliorare la leggibilità e ridurre l’affaticamento visivo. & \\
\hline
\textbf{RF-02} & L’estensione deve supportare il cambio della lingua dell’interfaccia dall’italiano all’inglese. & \\
\hline
\textbf{RF-03} & L’estensione deve mantenere la cronologia delle interazioni con l’IA nella sessione corrente. & \\
\hline
\end{longtable}
\end{footnotesize}



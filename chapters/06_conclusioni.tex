\chapter{Conclusioni}
\label{chap:conclusioni}
\noindent Al termine dello stage, si può affermare che l’obiettivo principale del progetto è stato in parte raggiunto. È stata infatti realizzata un’estensione web, denominata \textit{SviluppAbile}, capace di analizzare le pagine web e assistere gli sviluppatori nell’individuazione e correzione degli errori di accessibilità. L’attività ha richiesto più fasi, tra cui la progettazione dell’architettura dell’estensione, l’implementazione delle diverse funzionalità ed una consistente fase di testing finale. \\Particolare attenzione è stata posta all’integrazione del \glslink{chatbotg}{chatbot}, concepito per fornire spiegazioni chiare e in linguaggio naturale, così da rendere i contenuti accessibili anche a chi non ha conoscenze avanzate delle linee guida \acrshort{wcag} per l'accessiilità web.

\section{Raggiungimento degli obiettivi}
\noindent Il progetto SviluppAbile ha raggiunto pienamente l’obiettivo primario di realizzare uno strumento interattivo per il supporto allo sviluppo accessibile. Sono state implementate le funzionalità principali: la visualizzazione del codice sorgente delle pagine web e la possibilità di interagire con un \glslink{chatbotg}{chatbot} integrato per analizzare la pagina web, ricevere chiarimenti e suggerimenti personalizzati.\\
Un ulteriore traguardo è stato il completamento della modalità guidata, che affianca all’analisi automatica un ambiente di sviluppo interattivo, composto da tre pannelli, in cui l’utente può sperimentare correzioni assistite del codice e scaricare direttamente i blocchi proposti in formato \acrshort{html}. Questa funzionalità ha permesso di unire l’aspetto correttivo con quello formativo, rendendo lo strumento utile sia come validatore che come supporto didattico.\\
L’architettura basata esclusivamente su tecnologie web standard, conforme a Manifest V3, ha garantito la compatibilità con i \glslink{browserg}{browser} moderni e la facilità di distribuzione dell’estensione. 
In sintesi, il sistema ha dimostrato di soddisfare gli obiettivi stabiliti: fornire uno strumento intuitivo, accessibile e formativo, capace di supportare in tempo reale lo sviluppatore nella produzione di pagine web più accessibili.\\
Per quanto riguarda invece la qualità e la completa correttezza delle risposte e del codice generato, è emerso che l’\acrshort{ia} talvolta segnala errori inesistenti o, al contrario, non rileva alcune criticità reali. Inoltre il codice proposto non sempre risulta pienamente conforme alle best-practice in materia di accessibilità. Pertanto, il miglioramento non riguarda l’estensione in sé, bensì la scelta del modello di \glslink{iag}{intelligenza artificiale} e il suo continuo affinamento.\\
In particolare, si è reso necessario identificare la versione di Ollama più congrua per le esigenze del progetto: inizialmente erano stati avviati test utilizzando sia il PC personale (versione Mistral) sia quello dell’università (versione LLaMA 3.1:8B). Successivamente è stata valutata la versione 3.2:3B, che mostrava performance migliori, ma le risposte tendevano a contenere un maggior numero di errori di formulazione e grammaticali, oltre a imprecisioni nel contenuto; per questi motivi, si è scelto di adottare per tutti i test riportati in questa tesi la versione 3.1:8b, che garantisce un equilibrio più stabile tra accuratezza e affidabilità.

\subsection{Misurazione quantitativa dei requisiti soddisfatti}
\noindent Il confronto con i requisiti inizialmente definiti (vedi paragrafo \ref{sec:req}) evidenzia quanto segue:
\begin{itemize}
    \item Requisiti obbligatori: 6/6 implementati (100\%)
    \item Requisiti desiderabili: 4/4 soddisfatti (100\%)
    \item Requisiti facoltativi: 1/3 rispettati (33\%)
\end{itemize}
Quindi il tasso di completamento complessivo è di 11/13 requisiti (84\%).

\section{Sviluppi futuri}
\noindent L’estensione web \textit{SviluppAbile} costituisce una base su cui innestare ulteriori miglioramenti: grazie alla struttura modulare, sarà possibile introdurre nuove funzionalità senza compromettere la stabilità del sistema.\\
Un primo ambito di sviluppo riguarda l’integrazione di modelli di \glslink{iag}{intelligenza artificiale} più avanzati (come \glslink{ChatGPTg}{ChatGPT} o sistemi analoghi), in grado di fornire risposte più precise, complete e soprattutto più rapide. Questo permetterebbe di aumentare l’affidabilità del \glslink{chatbotg}{chatbot} e di garantire un supporto sempre più accurato agli sviluppatori.\\
Dal punto di vista dell’interfaccia utente, potrebbe essere introdotta una scansione automatica iniziale del \acrshort{dom}, che mostri fin da subito eventuali errori in un pannello dedicato, prendendo spunto da strumenti consolidati come \textit{WAVE}.\\
\\
Un’evoluzione significativa riguarda inoltre l’integrazione di tecniche di \textit{Computer Vision}, che consentirebbero di estendere l’analisi anche agli aspetti visivi della pagina, come il contrasto cromatico, la leggibilità dei testi e le proporzioni degli elementi grafici.\\
Altri sviluppi possibili includono il collegamento con validatori esterni (come TV, WAVE e Lighthouse), l’introduzione di un versionamento dei suggerimenti generati per facilitare il confronto tra diverse soluzioni e l’ottimizzazione delle prestazioni tramite caching o caricamento progressivo del \acrshort{dom}.
In prospettiva, \textit{SviluppAbile} potrebbe evolvere in un vero e proprio laboratorio per l’accessibilità, capace di coniugare analisi automatica, supporto interattivo e verifica visiva, offrendo agli sviluppatori uno strumento ancora più completo e formativo.

\section{Consuntivo finale}
\noindent Lo sviluppo dell’estensione \textit{SviluppAbile} ha richiesto un totale di 300 ore, in linea con il monte ore preventivato. Il tempo è stato principalmente suddiviso tra lo studio del linguaggio JavaScript e l’apprendimento dello sviluppo di estensioni per \glslink{browserg}{browser}, l’integrazione dell'\glslink{iag}{intelligenza artificiale} per l’analisi del \acrshort{dom}, il ripasso delle best-practice per l'accessibilità web e le attività finali di test per verificare il corretto funzionamento e l’affidabilità dell’estensione. 

\section{Competenze acquisite}
\noindent Il progetto di stage ha rappresentato un’importante occasione di crescita tecnica e professionale, permettendo di acquisire competenze trasversali fondamentali. Dal punto di vista dello sviluppo web, l’esperienza ha consolidato la mia conoscenza di \acrshort{html}, sia nella struttura delle pagine che nella gestione degli elementi interattivi, e delle tecniche per produrre codice accessibile e semanticamente corretto. L’uso di strumenti di validazione del codice, come Total Validator, ha permesso di comprendere a fondo le problematiche legate all’accessibilità e di applicare concretamente le linee guida \acrshort{wcag} durante lo sviluppo.\\
Particolare attenzione è stata posta all’integrazione di funzionalità complesse in un ambiente modulare e compatibile con Manifest V3, comprendendo lo sviluppo di pannelli interattivi, la comunicazione tra \glslink{scriptg}{script} di background e content \glslink{scriptg}{script} e l’uso iniziale delle \acrshort{api} di Chrome. L’implementazione del \glslink{chatbotg}{chatbot} ha richiesto la selezione di un’\acrshort{ia} utilizzabile, la cui scelta è ricaduta su Ollama, nonché la generazione dinamica di domande suggerite, il filtraggio delle risposte prodotte e la sincronizzazione con il codice visualizzato nei pannelli.\\
Il progetto ha avuto un carattere sperimentale, quindi non tutte le funzionalità da sviluppare erano definite fin dall’inizio; di conseguenza, non tutte le soluzioni ipotizzate sono state realizzabili, soprattutto a causa della complessità tecnica e dei vincoli di tempo. Questa esperienza ha comunque permesso di sviluppare capacità di problem solving, adattamento e gestione autonoma delle priorità, affrontando le difficoltà tipiche della sperimentazione.\\
Infine, il progetto ha permesso di sviluppare un approccio critico all’accessibilità digitale, comprendendo a fondo le linee guida \acrshort{wcag} e le strategie per guidare gli sviluppatori nel miglioramento della qualità dei propri siti. 
\\In sintesi, l’esperienza ha combinato conoscenze teoriche e pratiche, fornendo strumenti concreti per la realizzazione di applicazioni web interattive, performanti e inclusive.

\section{Valutazione personale}
\noindent Lo svolgimento di questo progetto di stage ha rappresentato la mia prima vera esperienza di ricerca, un’occasione preziosa che mi ha permesso di andare oltre la semplice applicazione delle conoscenze acquisite durante il percorso universitario. Da un lato, ho potuto consolidare competenze tecniche già note e sperimentarne di nuove, dall’altro ho avuto l’opportunità di affrontare problemi a me sconosciuti con un approccio critico e metodologico, tipico delle attività di ricerca. Particolarmente stimolante è stata la possibilità di coniugare aspetti teorici, come i principi di accessibilità e le normative \acrshort{wcag}, con aspetti pratici legati alla realizzazione di uno strumento effettivamente utile agli sviluppatori. Ho trovato gratificante anche la libertà progettuale concessa, che mi ha spinto a sperimentare soluzioni non sempre funzionanti e a trasformare le difficoltà incontrate in momenti di crescita personale e professionale. \\
In conclusione, considero questa esperienza di stage un passaggio fondamentale del mio percorso formativo, tanto da rivalutare la possibilità di proseguire con una laurea magistrale (in \textit{Intelligenza Artificiale} oppure in \textit{Programming Languages, Systems and Algorithms}), anche se a tempo parziale.